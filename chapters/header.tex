\providecommand{\keywords}[1]
{
\small
\textbf{\textit{Keywords---}} #1
}

\titulo{Recognition of \RL-partite Graphs}
\instituicao{UNIVERSIDADE FEDERAL FLUMINENSE}
\autor{Matheus S. D'Andrea Alves}
\orientador{Uéverton dos Santos Souza}
\local{Niterói}
\comentario{Dissertação de Mestrado apresentada ao Programa de P\'{o}s-Gradua\c{c}\~{a}o em Computa\c{c}\~{a}o da \mbox{Universidade} Federal Fluminense
como requisito parcial para a obten\c{c}\~{a}o do Grau de \mbox{Mestre em Computa\c{c}\~{a}o}. \'{A}rea de concentra\c{c}\~{a}o: \mbox{Algoritmos e Otimização}} %preencha com a sua area de concentracao
\data{2020}
\capa
\folhaderosto

\begin{abstract}
    The complexity to recognize if a graph has an \RL-partition, i.e. if it can be partitioned into $r$ cliques and $\ell$ independent sets, is well defined\cite{brand-84}.
    However, as we will demonstrate, the literature-stabilished values for those on the \textit{P} class can be improved.
    The following work provides a set of strategies and algortihms that pushes the previous results for the $(2,1)$-partite (from $n^4$ to $n*m$),
    $(1,2)$-partite (from $n^4$ to $n*m$) and $(2,2)$-partite (from $n^{12}$ to $n^{2}*m$) recognition.

    \keywords{\RL-graphs, \RL-partitions}
\end{abstract}

\listoffigures
\listoftables
\tableofcontents
