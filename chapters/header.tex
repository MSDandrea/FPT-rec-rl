\begin{document}
\providecommand{\keywords}[1]
{
  \small
  \textbf{\textit{Keywords---}} #1
}

\title{Recognition of \RL-partite Graphs}
\author{Matheus S. D'Andrea Alves}
\date{2019}
\imprimircapa

\begin{resumo}
  The complexity to recognize if a graph has an \RL-partition, i.e. if it can be partitioned into $r$ cliques and $\ell$ independent sets, is well defined\cite{brand-84}. However, as we will demonstrate, the literature-stabilished values for those on the \textit{P} class can be improved. The following work provides a set of strategies and algortihms that pushes the previous results for the $(2,1)$-partite (from $n^4$ to $n*m$), $(1,2)$-partite (from $n^4$ to $n*m$) and $(2,2)$-partite (from $n^{12}$ to $n^{2}*m$) recognition.

  \keywords{\RL-graphs, \RL-partitions}
\end{resumo}

\listoffigures*
\listoftables*
\tableofcontents
