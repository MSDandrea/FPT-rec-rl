\chapter{Our strategy}\label{ch:os}


\section{Finding the 3,1 partition}\label{sec:3,1-partition}

As seen on the Brandstädt's algorithm every vertex of a 2,1 partition must at least be compliance to one of the neighborhood rules:

\begin{itemize}
    \item (N1) $N(v)$ induces a split graph.
    \item (N2) Or, $\bar{N}(v)$ should induce a bipartite graph.
\end{itemize}

If every vertex match exactly one of these rules, then the Graph is known to be a 2,1 Graph.
On the other hand if any vertex does not match N1 or N2 then the Graph is bound to not have a 2,1 partition.
However, when a vertex matches both N1 and N2 the graph could have a 2,1 partition but clearly has a 3,1 partition as the union of the induced split of it's open neighborhood, and the bipartite induced by its complement describe such partition.

Therefore, the following algorithm can be declared.
%% @formatter:off
\begin{code}{Get 2-1-Partition}
def Get_21(Graph G) 2-1-Partition
  cl = new Graph #intended clique
  bi = new Graph #intended bipartite
  for v in V(G) do # O(n)
      N = |$N(v)$| # open neighborhood of v
      K = |$\bar{N}(v)$| # V(G) - N
      if N.is_Split() then # O(m)
        bi.Add(v)
      if K.is_Bipartite() then # O(m)
        cl.Add(v)
      if (!bi.contains(v) and !cl.contains(v)) then
       return None; # G $\notin$ (2,1)
  end # O(n*m)
  if |$cl.V \cap bi.V = \emptyset$| then
    if (!cl.is_Clique() or !bi._is_Bipatite()) then
      return None; # G $\notin$ (2,1)
    else return new 2-1-patition{
      |$C$|: cl
      |$I_1$|: bi.|$I_1$|,
      |$I_2$|: bi.|$I_2$|
    }
  else
    using any |$v \in cl.V \cap bi.V$|
    N = |$N(v)$| # open neighborhood of v
    K = |$\bar{N}(v)$| # V(G) - N
    tr = new 3-1-Partition{
          |$C$|: N.Clique,
          |$I_1$|: N.Independent,
          |$I_2$|: K.|$I_1$|,
          |$I_3$|: K.|$I_2$|,
    }
    return 3-1-to-2-1(tr)
  end
end
\end{code}
%% @formatter:on


\section{Odd Cycle transversal}\label{sec:oct}

The Odd Cycle transversal can be described as the problem of finding a subset of vertices of a graph such that when removed from said graph makes it a bipartite graph.
This problem is known to be NP-Complete, however, Reed, Smith and Vetta presented a fixed-parameter tractable algorithm for this problem.
Such algorithm obtains the desirable subset running in time $3^{k}n^{\mathcal{O}(1)}$ \cite{reed-04}

Is imperative we further analyze this algorithm, as it's strategy will be relevant for our next steps.

\subsection{The iterative compression technique}\label{subsec:the-iterative-compression-technique}

The solution developed in~\cite{reed-04} is the base for the now called Iterative Compression technique;
A compression can be described as following:

\begin{itemize}
    \item Assume there is a non-optimal solution $S$ of size $k+1$ for the problem $A$
    \item Using the given solution as input find a solution for the Disjoint-$A$ problem, i.e. find a smaller solution $X$ of size $k$ in $G-P_S$ where $P_S$ is every subset of $S$ of size $ \leq |S| - |X|$.
\end{itemize}


It's trivial to notice that this steps can be applied as many times as needed to reduce the non-optimal solution to the desired size solution,
or to declare that such solution does not exist, therefore the iterative part of the technique name.
Some solutions for the problems \emph{Vertex Cover, Feedback Vertex Set, Feedback Vertex Set in Tournaments} and the solution we used can be found in~\cite{cygan-2015}

\subsection{A FPT Solution of Odd Cycle Transversal}\label{subsec:fpt-solution-oct}

As we described at \ref{sec:oct}, $X$ is an odd cycle transversal of $G$ if the graph $G-X$ is bipartite.
In order to use the steps declared on \ref{subsec:the-iterative-compression-technique} we must define Disjoint-Odd Cycle Transversal.

The input of this problem is $G$ an undirected graph, an odd cycle transversal $W$ of size $k+1$ and a positive integer $k$.
The objective is to find an odd cycle transversal $X \subseteq V(G) \setminus W$ of size at most $k$ or to conclude that $X$ does not exist.
The presented algorithm gives a solution in time $2^{k}n^{\mathcal{O}(1)}$.

The algorithm For Disjoint-Odd Cycle Transversal will use the following annotated problem as a subroutine.
In the \emph{Annotated Bipartite Coloring} problem, we are given an undirected graph $G$, two sets $B_1,B_2 \subseteq V(G)$ and a positive integer $k$.
The goal is to find a subset $X$ of $V(G)$ with size at most $k$ such that $G-X$ has a coloring $f : V(G) \setminus X \rightarrow {1,2}$ where $f(v) \neq f(u)$ for every edge $(u,v)$
and $f(v) = i$ whenever $v \in B_i \setminus X$, i.e. if a vertex $v$ is originally in $B_1$ the color of it in the result shall be $f(v)=1$.

\textbf{Annotated Bipartite Coloring algorithm.}

Let $(G,B_1,B_2,k)$ be an instance of Annotated Bipartite Coloring.
We can view the vertices of $B_1$ and $B_2$ as precolored vertices from a function $l$, i.e. if $v \in B_i$ then $l(v)=i$.
Is important to notice that $B_1$ adn $B_2$ are not disjoint, meaning any vertex can have both colors, also, $l$ does not necessarily provide a proper coloring,
allowing the coloring $l(v)=l(u)$ for an edge $(v,u)$.

We want to find a set $X$ of size at most $k$ such that in graph $G-X$ there is a proper 2-coloring $f$ extending precolored vertices.
To find such coloring we proceed as follows.

Fix an arbitrary \emph{proper} 2-coloring $f^*$ of $G$ issuing the $B_1^*,B_2^*$ partitions, such coloring exists as $G$ is bipartite.
Let $C$ be the following set $C := (B_1 \cap B_2^*) \cup (B_2 \cap B_1^*)$, every vertex in $C$ should either be included in $X$ or have "changed" colors,
and similarly each vertex in $R := (B_1 \cap B_1^*) \cup (B_2 \cap B_2^*)$ have retained its color or is a vertex in $X$.
The following lemma helps us solve the annotated problem.

\begin{lemma}\label{lemma:annotated-bipartite-coloring}
  Let $G$ be a bipartite graph and $f^*$ an arbitrary proper coloring of $G$.
  Then set $X$ is a solution to the Annotated Bipartite Coloring problem if and only if $X$ separates $C$ and $R$.
  Furthermore, provided that such $X$ of size $k$ exists it can be found in time \O{k(n*m)}.
\end{lemma}
\begin{proof}
  In a proper 2-coloring $f$ of $G-X$, each vertex of $V(G) \setminus X$ either keeps the same color ($f(v)=f^*(v)$) or changes it colors ($f(v) \neq f^*(v)$).
  Moreover, every vertex in a connected component of $G$ must make the same decision in order to maintain the proper coloration, i.e. for every edge $(u,v)$ if $v$ changed then $u$ has to change.
  Therefore, no connected component of $G \setminus X$ may contain a vertex from both $R$ and $C$.
  Consequently, $X$ has to be a separating set of $R$ and $C$.

  Let $X$ be a set separating $R$ and $C$, define $f$ as follows: First set $f=f^*$ and then flip the coloring of those components go $G-X$ that contain at least one vertex of $C \setminus X$.
  As $X$ is a separating set no vertex of $R$ is flipped and thus we have a proper 2-coloring.

  To find a  vertex set of size at most $k$ separating $R$ from $C$ one can use the classic max-flox min-cut techniques,
  e.g. by applying $k$ times the iteration of the Ford-Fulkerson algorithm obtaining the promised time bound. %adicionar o algoritmo de ford fulkerson no apendice?
\end{proof}

Now we can apply this solution to the Annotated Bipartite Coloring to solve ou Odd Cycle transversal problem.
In order to do so, recall that we have to solve the Disjoint version of this problem, i.e. given  


\section{The critical paths}\label{sec:tcp}


\section{Finding the viable movements on the critical paths}\label{sec:path-movements}


\section{The Algorithm}

\subsection{Pseudo-Code}

%% %% @formatter:off
\begin{code}{3-1-to-2-1}
def 3-1-to-2-1(Graph-3-1 G)
  tri = |$G.I_1 \cup G.I_2 \cup G.I_3$|
  oct = Odd-Cycle-Transversal(tri,3)
  case oct is None then
    return None; # There's more then 3 vertices in the tripartite that should move.
  case oct = 1 then
    for v in tri.v do # O(n)
      inter = |$\bar{N}(v) \cap G.C$|
      if size(inter) > 2 then
        break loop;
      cl = G.C - inter + v
      bi = tri - v + inter
      if (cl.is_Clique() and bi.is_Bipartite()) then # O(m)
        return new 2-1-patition{
          |$C$|: cl
          |$I_1$|: bi.|$I_1$|,
          |$I_2$|: bi.|$I_2$|
        }
      end
    end
    return None; # There's no vertex that can be moved.
  case oct = 2 then
    return Transform2(tri,G.C,G)
  case oct = 3 then
    for v in oct do
      inter = |$\bar{N}(v) \cap G.C$|
      if size(inter) > 2 then
        break loop;
      cl = G.C - inter + v
      neoTri = tri - v + inter
      result = Transform2(neoTri,cl,G)
      if result is not None then
        return result;
    end
    return None;
end
\end{code}
%% @formatter:on

%% @formatter:off
\begin{code}{Transform2}
def Transform2(Tripartite,clique,G)
  |$P_1$|, |$P_2$| = CriticalPaths(Tripartite)
  # Table
  # rank \ aux
  rank1 = Table[Label][Int][Int]
  for v in |$V(G[P_1])$| do # Rank 1
    max_out = 0
    for out in |$E^+(v) \in G[P_1]$| do
      max_out = Max(out,max_out)
    end # max_out is the closest vertex to T that can be reached by an out edge
    previous = v.prev
    rank = v.position
    if previous is not None then
      # rank is either v position or the foremost
      # vertex that can be reached by the previous vertex
      rank = Max(rank, rank1[previous][1])
      # max_out is the max between the farthest
      # vertex that can be reached by v or by the previous vertex
      max_out = Max(max_out,rank1[previous][1])
    end
    rank1[v][0] = rank
    rank1[v][1] = max_out
  end
  rank2 = Table[Label][Int][Int]
  for u in |$V(G[P_2])$| do # Rank 2
    max_out = 0
    for out in |$E^+(u) \in G[P_2]$| do
      max_out = Max(out,max_out)
    end # max_out is the closest vertex to T that can be reached by an out edge
    previous = u.prev
    rank = u.position
    if previous is not None then
      # rank is either u position or the foremost
      # vertex that can be reached by the previous vertex
      rank = Max(rank, rank2[previous][1])
      # max_out is the max between the farthest
      # vertex that can be reached by u or by the previous vertex
      max_out = Max(max_out,rank2[previous][1])
    end
    rank2[u][0] = rank
    rank2[u][1] = max_out
  end
  for v in |$V(G[P_2])$| # Rank 1.2
    max_out = 0
    for out in |$E^+(v) \in G[P_1 \cup v]$| do
      max_out = Max(out,max_out)
    end # max_out is the closest vertex to T that can be reached by an out edge
    previous = v.prev
    rank = v.position
    if previous is not None then
      # rank is either v position or the foremost
      # vertex that can be reached by the previous vertex in $P_1$
      rank = Max(rank, rank1[previous][1])
      # max_out is the max between the farthest
      # vertex that can be reached by v or by the previous vertex
      max_out = Max(max_out,rank1[previous][1])
    end
    rank1[v][0] = rank
    rank1[v][1] = max_out
  end
  for u in |$V(G[P_1])$| # Rank 2.1
    max_out = 0
    for out in |$E^+(u) \in G[P_2 \cup u]$| do
      max_out = Max(out,max_out)
    end # max_out is the closest vertex to T that can be reached by an out edge
    previous = u.prev
    rank = u.position
    if previous is not None then
      # rank is either u position or the foremost
      # vertex that can be reached by the previous vertex in $P_2$
      rank = Max(rank, rank2[previous][1])
      # max_out is the max between the farthest
      # vertex that can be reached by v or by the previous vertex
      max_out = Max(max_out,rank2[previous][1])
    end
    rank2[v][0] = rank
    rank2[v][1] = max_out
  end
  for z in |$V(clique)$| do # for all vertex in the clique
    # take all edges on the tripartite that have an end on z
    for e(z,i) in |$E(Tripartite \cap N[z])$| do
      z.left_u = |$+\infty$|
      z.left_v = |$+\infty$|
      z.right_u = 0
      z.right_v = 0
      # Find which are the closest and farthest vertex in $P_1$ z can reach
      if i in |$V(G[P_1])$| then
        if i.position < z.left_u then
          z.left_u = i.position
        end
        if i.position > z.right_u then
          z.right_u = i.position
        end
      end
      # Find which are the closest and farthest vertex in $P_2$ z can reach
      if i in |$V(G[P_2])$| then
        if i.position < z.left_v then
          z.left_v = i.position
        end
        if i.position > z.right_v then
          z.right_v = i.position
        end
      end
    end
  end
  for |$e(u,v) \in E(G)$| where |$u \in V(G[P_1])$| and |$u \in V(G[P_1])$| do
    # let $\bar{N_k(v)}$ be the not-neighborhood of a vertex in the clique
    S = |$\bar{N_k(u)} \cap \bar{N_k(v)}$|
    if size(S) <=2 then
     if rank1[u][0] <= u.position then
       if rank2[v][0] <= v.position then
         if rank2[u][0] <= v.position and rank1[v][0] <= u.position then
           if size(S) = 0 then
             c = |$clique \cup \{u,v\}$|
             bi = Bipartition_Of(|$Tripartite \setminus \{u,v\}$|)
             return new 2-1-patition{
               |$C$|: c
               |$I_1$|: bi.|$I_1$|,
               |$I_2$|: bi.|$I_2$|
             }
           end
           max_right_u = 0
           max_right_v = 0
           min_left_u = |$+\infty$|
           min_left_v = |$+\infty$|
           c = |$clique \cup \{u,v\} \setminus S$|
           bi = Bipartition_Of(|$Tripartite \cup S \setminus \{u,v\}$|)
           h = new 2-1-patition{
               |$C$|: c
               |$I_1$|: bi.|$I_1$|,
               |$I_2$|: bi.|$I_2$|
             }
           for z in V(S) do
            if z.right_u > max_right_u then
              max_right_u = z.right_u
            end
            if z.right_v > max_right_v then
              max_right_v = z.right_v
            end
            if z.left_v < min_left_v then
              min_left_v = z.left_v
            end
            if z.left_u < min_left_u then
              min_left_u = z.left_u
            end
           end
           if max_right_u < u.position and max_right_v < v.position then
             return h;
           end
           if min_left_u > u.position and min_left_v > v.position then
             return h;
           end
         end
       end
     end
    end
  end
  return None;
end
\end{code}
%% @formatter:on

