\chapter{Our strategy}\label{ch:os}


\section{Finding the 3,1 partition}\label{sec:3,1-partition}

As seen on the Brandstädt's algorithm every vertex of a 2,1 partition must at least be compliance to one of the neighborhood rules:

\begin{itemize}
  \item (N1) $N(v)$ induces a split graph.
  \item (N2) Or, $\bar{N}(v)$ should induce a bipartite graph.
\end{itemize}

If every vertex match exactly one of these rules, then the Graph is known to be a 2,1 Graph.
On the other hand if any vertex does not match N1 or N2 then the Graph is bound to not have a 2,1 partition.
However, when a vertex matches both N1 and N2 the graph could have a 2,1 partition but clearly has a 3,1 partition as the union of the induced split of it's open neighborhood, and the bipartite induced by its complement describe such partition.

Therefore, the following algorithm can be declared.
\begin{code}{Get 2-1-Partition}
def Get_21(Graph G) 2-1-Partition
  cl = new Graph #intended clique
  bi = new Graph #intended bipartite 
  for v in V(G) do # O(n)
      N = |$N(v)$| # open neighborhood of v
      K = |$\bar{N}(v)$| # V(G) - N
      if N.is_Split() then # O(m)
        bi.Add(v)
      if K.is_Bipartite() then # O(m)
        cl.Add(v)
      if (!bi.contains(v) and !cl.contains(v)) then
       return None; # G $\notin$ (2,1)
  end # O(n*m)
  if |$cl.V \cap bi.V = \emptyset$| then
    if (!cl.is_Clique() or !bi._is_Bipatite()) then
      return None; # G $\notin$ (2,1)
    else return new 2-1-patition{
      |$C$|: cl
      |$I_1$|: bi.|$I_1$|,
      |$I_2$|: bi.|$I_2$|
    }
  else 
    using any |$v \in cl.V \cap bi.V$|
    N = |$N(v)$| # open neighborhood of v
    K = |$\bar{N}(v)$| # V(G) - N
    tr = new 3-1-Partition{
          |$C$|: N.Clique,
          |$I_1$|: N.Independent,
          |$I_2$|: K.|$I_1$|,
          |$I_3$|: K.|$I_2$|,
    }
    return 3-1-to-2-1(tr)
  end
end
\end{code}

\section{Odd Cycle transversal}\label{sec:oct}

The Odd Cycle transversal can be described as the problem of finding a subset of vertices of a graph such that when removed from said graph makes it a bipartite graph.
This problem is known to be NP-Complete, however, Reed, Smith and Vetta presented a fixed-parameter tractable algorithm for this problem.
Such algorithm obtain the desirable subset running in time $3^kn^{\mathcal{O}(1)}$ \cite{reed-04}

Is imperative we further analyze this algorithm, as it's strategy will be relevant for our next steps.

\section{The critical paths}\label{sec:tcp}


\section{Finding the viable movements on the critical paths}\label{sec:path-movements}


\section{The Algorithm}

\subsection{Pseudo-Code}

\begin{code}{3-1-to-2-1}
def 3-1-to-2-1(Graph-3-1 G)
  tri = |$G.I_1 \cup G.I_2 \cup G.I_3$| 
  oct = Odd-Cycle-Transversal(tri,3)
  case oct is None then
    return None; # There's more then 3 vertices in the tripartite that should move.
  case oct = 1 then
    for v in tri.v do # O(n)
      inter = |$\bar{N}(v) \cap G.C$|
      if size(inter) > 2 then
        break loop; 
      cl = G.C - inter + v
      bi = tri - v + inter
      if (cl.is_Clique() and bi.is_Bipartite()) then # O(m)
        return new 2-1-patition{
          |$C$|: cl
          |$I_1$|: bi.|$I_1$|,
          |$I_2$|: bi.|$I_2$|
        }
      end
    end
    return None; # There's no vertex that can be moved.
  case oct = 2 then
    return Transform2(tri,G.C,G)
  case oct = 3 then
    for v in oct do
      inter = |$\bar{N}(v) \cap G.C$|
      if size(inter) > 2 then
        break loop; 
      cl = G.C - inter + v
      neoTri = tri - v + inter
      result = Transform2(neoTri,cl,G)
      if result is not None then
        return result;
    end
    return None; 
end
\end{code}

% \begin{code}{CriticalPaths}
% def CriticalPaths(Tripartite)
% \end{code}

\begin{code}{Transform2}
def Transform2(Tripartite,clique,G)
  |$P_1$|, |$P_2$| = CriticalPaths(Tripartite)
  # Table
  # rank \ aux
  rank1 = Table[Label][Int][Int]
  for v in |$V(G[P_1])$| do # Rank 1
    max_out = 0
    for out in |$E^+(v) \in G[P_1]$| do
      max_out = Max(out,max_out)
    end # max_out is the closest vertex to T that can be reached by an out edge
    previous = v.prev
    rank = v.position
    if previous is not None then
      # rank is either v position or the foremost 
      # vertex that can be reached by the previous vertex
      rank = Max(rank, rank1[previous][1]) 
      # max_out is the max between the farthest
      # vertex that can be reached by v or by the previous vertex
      max_out = Max(max_out,rank1[previous][1])  
    end
    rank1[v][0] = rank
    rank1[v][1] = max_out
  end
  rank2 = Table[Label][Int][Int]
  for u in |$V(G[P_2])$| do # Rank 2
    max_out = 0
    for out in |$E^+(u) \in G[P_2]$| do
      max_out = Max(out,max_out)
    end # max_out is the closest vertex to T that can be reached by an out edge
    previous = u.prev
    rank = u.position
    if previous is not None then
      # rank is either u position or the foremost 
      # vertex that can be reached by the previous vertex
      rank = Max(rank, rank2[previous][1]) 
      # max_out is the max between the farthest
      # vertex that can be reached by u or by the previous vertex 
      max_out = Max(max_out,rank2[previous][1]) 
    end
    rank2[u][0] = rank
    rank2[u][1] = max_out
  end
  for v in |$V(G[P_2])$| # Rank 1.2
    max_out = 0
    for out in |$E^+(v) \in G[P_1 \cup v]$| do
      max_out = Max(out,max_out)
    end # max_out is the closest vertex to T that can be reached by an out edge
    previous = v.prev
    rank = v.position
    if previous is not None then
      # rank is either v position or the foremost 
      # vertex that can be reached by the previous vertex in $P_1$
      rank = Max(rank, rank1[previous][1]) 
      # max_out is the max between the farthest
      # vertex that can be reached by v or by the previous vertex
      max_out = Max(max_out,rank1[previous][1])  
    end
    rank1[v][0] = rank
    rank1[v][1] = max_out
  end
  for u in |$V(G[P_1])$| # Rank 2.1
    max_out = 0
    for out in |$E^+(u) \in G[P_2 \cup u]$| do
      max_out = Max(out,max_out)
    end # max_out is the closest vertex to T that can be reached by an out edge
    previous = u.prev
    rank = u.position
    if previous is not None then
      # rank is either u position or the foremost 
      # vertex that can be reached by the previous vertex in $P_2$
      rank = Max(rank, rank2[previous][1]) 
      # max_out is the max between the farthest
      # vertex that can be reached by v or by the previous vertex
      max_out = Max(max_out,rank2[previous][1])  
    end
    rank2[v][0] = rank
    rank2[v][1] = max_out
  end
  for z in |$V(clique)$| do # for all vertex in the clique
    # take all edges on the tripartite that have an end on z
    for e(z,i) in |$E(Tripartite \cap N[z])$| do 
      z.left_u = |$+\infty$|
      z.left_v = |$+\infty$|
      z.right_u = 0
      z.right_v = 0
      # Find which are the closest and farthest vertex in $P_1$ z can reach
      if i in |$V(G[P_1])$| then 
        if i.position < z.left_u then
          z.left_u = i.position
        end
        if i.position > z.right_u then
          z.right_u = i.position
        end
      end
      # Find which are the closest and farthest vertex in $P_2$ z can reach
      if i in |$V(G[P_2])$| then 
        if i.position < z.left_v then
          z.left_v = i.position
        end
        if i.position > z.right_v then
          z.right_v = i.position
        end
      end  
    end
  end
  for |$e(u,v) \in E(G)$| where |$u \in V(G[P_1])$| and |$u \in V(G[P_1])$| do
    # let $\bar{N_k(v)}$ be the not-neighborhood of a vertex in the clique
    S = |$\bar{N_k(u)} \cap \bar{N_k(v)}$|
    if size(S) <=2 then
     if rank1[u][0] <= u.position then
       if rank2[v][0] <= v.position then
         if rank2[u][0] <= v.position and rank1[v][0] <= u.position then
           if size(S) = 0 then
             c = |$clique \cup \{u,v\}$|
             bi = Bipartition_Of(|$Tripartite \setminus \{u,v\}$|)
             return new 2-1-patition{
               |$C$|: c
               |$I_1$|: bi.|$I_1$|,
               |$I_2$|: bi.|$I_2$|
             }
           end
           max_right_u = 0
           max_right_v = 0
           min_left_u = |$+\infty$|
           min_left_v = |$+\infty$|
           c = |$clique \cup \{u,v\} \setminus S$|
           bi = Bipartition_Of(|$Tripartite \cup S \setminus \{u,v\}$|)
           h = new 2-1-patition{
               |$C$|: c
               |$I_1$|: bi.|$I_1$|,
               |$I_2$|: bi.|$I_2$|
             }
           for z in V(S) do
            if z.right_u > max_right_u then
              max_right_u = z.right_u
            end
            if z.right_v > max_right_v then
              max_right_v = z.right_v
            end
            if z.left_v < min_left_v then
              min_left_v = z.left_v
            end
            if z.left_u < min_left_u then
              min_left_u = z.left_u
            end
           end
           if max_right_u < u.position and max_right_v < v.position then
             return h;
           end
           if min_left_u > u.position and min_left_v > v.position then
             return h;
           end
         end
       end
     end
    end    
  end
  return None;
end
\end{code}

\subsection{Correctness}
