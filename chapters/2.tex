\chapter{Our strategy}\label{ch:os}


\section{Finding the 3,1 partition}\label{sec:3,1-partition}

As seen on the Brandstädt's algorithm every vertex of a 2,1 partition must at least be compliance to one of the neighborhood rules:

\begin{itemize}
    \item (N1) $N(v)$ induces a split graph.
    \item (N2) Or, $\bar{N}(v)$ should induce a bipartite graph.
\end{itemize}

If every vertex match exactly one of these rules, then the Graph is known to be a 2,1 Graph.
On the other hand if any vertex does not match N1 or N2 then the Graph is bound to not have a 2,1 partition.
However, when a vertex matches both N1 and N2 the graph could have a 2,1 partition but clearly has a 3,1 partition as the union of the induced split of it's open neighborhood, and the bipartite induced by its complement describe such partition.

Therefore, the following algorithm can be declared.
%% @formatter:off
\begin{code}{Get 2-1-Partition}
def Get_21(Graph G) 2-1-Partition
  cl = new Graph #intended clique
  bi = new Graph #intended bipartite
  for v in V(G) do # O(n)
      N = |$N(v)$| # open neighborhood of v
      K = |$\bar{N}(v)$| # V(G) - N
      if N.is_Split() then # O(m)
        bi.Add(v)
      if K.is_Bipartite() then # O(m)
        cl.Add(v)
      if (!bi.contains(v) and !cl.contains(v)) then
       return None; # G $\notin$ (2,1)
  end # O(n*m)
  if |$cl.V \cap bi.V = \emptyset$| then
    if (!cl.is_Clique() or !bi._is_Bipatite()) then
      return None; # G $\notin$ (2,1)
    else return new 2-1-patition{
      |$C$|: cl
      |$I_1$|: bi.|$I_1$|,
      |$I_2$|: bi.|$I_2$|
    }
  else
    using any |$v \in cl.V \cap bi.V$|
    N = |$N(v)$| # open neighborhood of v
    K = |$\bar{N}(v)$| # V(G) - N
    tr = new 3-1-Partition{
          |$C$|: N.Clique,
          |$I_1$|: N.Independent,
          |$I_2$|: K.|$I_1$|,
          |$I_3$|: K.|$I_2$|,
    }
    return 3-1-to-2-1(tr)
  end
end
\end{code}
%% @formatter:on


\section{Odd Cycle transversal}\label{sec:oct}

The Odd Cycle transversal can be described as the problem of finding a subset of vertices of a graph such that when removed from said graph makes it a bipartite graph.
This problem is known to be NP-Complete, however, Reed, Smith and Vetta presented a fixed-parameter tractable algorithm for this problem.
Such algorithm obtains the desirable subset running in time $3^{k}n^{\mathcal{O}(1)}$ \cite{reed-04}

Is imperative we further analyze this algorithm, as it's strategy will be relevant for our next steps.

\subsection{The iterative compression technique}\label{subsec:the-iterative-compression-technique}

The solution developed in~\cite{reed-04} is the base for the now called Iterative Compression technique;
A $A$-Compression problem can be solved as following:

\begin{itemize}
    \item Assume there is a non-optimal solution $S$ of size $k+1$ for the problem $A$
    \item Using the given solution as input find a solution for the Disjoint-$A$ problem, i.e. find a smaller solution $X$ of size $k$ in $G-P_S$ where $P_S$ is every subset of $S$ of size $ \leq |S| - |X|$.
\end{itemize}

This steps raises the following statements:

\begin{tcolorbox}
  \begin{itemize}
    \item If there exists an algorithm solving Disjoint-$A$ in time $g(k)n^c$ then there exists an algorithm solving $A$-Compression in time $$ \sum_{i=0}^k \binom{k+1}{i}g(k-i)n^{\mathcal{O}(1)}$$
          In particular, if $g(k)=\alpha^k$ then $A$-Compression can be solved in $(1+\alpha)^{k}n^{\mathcal{O}(1)}$
    \item If there exists an algorithm solving $A$-Compression in time $f(k)n^c$, then there exists an algorithm solving problem $A$ in time \O{f(k)n^{c+1}}.
  \end{itemize}
\end{tcolorbox}
Further explanations, some solutions for the problems \emph{Vertex Cover, Feedback Vertex Set, Feedback Vertex Set in Tournaments} and the solution we used can be found in~\cite{reed-04,cygan-2015}

\subsection{A FPT Solution of Odd Cycle Transversal}\label{subsec:fpt-solution-oct}

As we described at \ref{sec:oct}, $X$ is an odd cycle transversal of $G$ if the graph $G-X$ is bipartite.
In order to use the steps declared on \ref{subsec:the-iterative-compression-technique} we must define Disjoint-Odd Cycle Transversal.

The input of this problem is $G$ an undirected graph, an odd cycle transversal $W$ of size $k+1$ and a positive integer $k$.
The objective is to find an odd cycle transversal $X \subseteq V(G) \setminus W$ of size at most $k$ or to conclude that $X$ does not exist.
The presented algorithm gives a solution in time $2^{k}n^{\mathcal{O}(1)}$.

The algorithm For Disjoint-Odd Cycle Transversal will use the following annotated problem as a subroutine.
In the \emph{Annotated Bipartite Coloring} problem, we are given an undirected graph $G$, two sets $B_1,B_2 \subseteq V(G)$ and a positive integer $k$.
The goal is to find a subset $X$ of $V(G)$ with size at most $k$ such that $G-X$ has a coloring $f : V(G) \setminus X \rightarrow {1,2}$ where $f(v) \neq f(u)$ for every edge $(u,v)$
and $f(v) = i$ whenever $v \in B_i \setminus X$, i.e. if a vertex $v$ is originally in $B_1$ the color of it in the result shall be $f(v)=1$.

\textbf{Annotated Bipartite Coloring algorithm.}

Let $(G,B_1,B_2,k)$ be an instance of Annotated Bipartite Coloring.
We can view the vertices of $B_1$ and $B_2$ as precolored vertices from a function $l$, i.e. if $v \in B_i$ then $l(v)=i$.
Is important to notice that $B_1$ adn $B_2$ are not disjoint, meaning any vertex can have both colors, also, $l$ does not necessarily provide a proper coloring,
allowing the coloring $l(v)=l(u)$ for an edge $(v,u)$.

We want to find a set $X$ of size at most $k$ such that in graph $G-X$ there is a proper 2-coloring $f$ extending precolored vertices.
To find such coloring we proceed as follows.

Fix an arbitrary \emph{proper} 2-coloring $f^*$ of $G$ issuing the $B_1^*,B_2^*$ partitions, such coloring exists as $G$ is bipartite.
Let $C$ be the following set $C := (B_1 \cap B_2^*) \cup (B_2 \cap B_1^*)$, every vertex in $C$ should either be included in $X$ or have "changed" colors,
and similarly each vertex in $R := (B_1 \cap B_1^*) \cup (B_2 \cap B_2^*)$ have retained its color or is a vertex in $X$.
The following lemma helps us solve the annotated problem.

\begin{lemma}\label{lemma:annotated-bipartite-coloring}
  Let $G$ be a bipartite graph and $f^*$ an arbitrary proper coloring of $G$.
  Then set $X$ is a solution to the Annotated Bipartite Coloring problem if and only if $X$ separates $C$ and $R$.
  Furthermore, provided that such $X$ of size $k$ exists it can be found in time \O{k(n*m)}.
\end{lemma}
\begin{proof}
  In a proper 2-coloring $f$ of $G-X$, each vertex of $V(G) \setminus X$ either keeps the same color ($f(v)=f^*(v)$) or changes it colors ($f(v) \neq f^*(v)$).
  Moreover, every vertex in a connected component of $G$ must make the same decision in order to maintain the proper coloration, i.e. for every edge $(u,v)$ if $v$ changed then $u$ has to change.
  Therefore, no connected component of $G \setminus X$ may contain a vertex from both $R$ and $C$.
  Consequently, $X$ has to be a separating set of $R$ and $C$.

  Let $X$ be a set separating $R$ and $C$, define $f$ as follows: First set $f=f^*$ and then flip the coloring of those components go $G-X$ that contain at least one vertex of $C \setminus X$.
  As $X$ is a separating set no vertex of $R$ is flipped and thus we have a proper 2-coloring.

  To find a  vertex set of size at most $k$ separating $R$ from $C$ one can use the classic max-flox min-cut techniques,
  e.g. by applying $k$ times the iteration of the Ford-Fulkerson algorithm obtaining the promised time bound. %adicionar o algoritmo de ford fulkerson no apendice?
\end{proof}

Now we can apply this solution to the Annotated Bipartite Coloring to solve ou Odd Cycle transversal problem.
In order to do so, recall that we have to solve the Disjoint version of this problem,
i.e. given  a Graph $G$, an integer $k$, and a set $W$ of $k+1$ vertices such that $G-W$ is bipartite.
The objective is to find a set $X \subseteq V(G) \setminus W$ of at most $k$ vertices such that $G-X$ is bipartite or to conclude that such set does not exists.
Towards we do as follows:

Every vertex $v \in W$ remains in $G-X$ therefore $G[W]$ is bipartite, as otherwise this is no instance. The main ideia is to \emph{guess} the bipartition of $W$ in $G-X$.
That is, we iterate over all proper 2-coloring $f_w: W \rightarrow \{1,2\}$ of $G[W]$ looking for a set $X \subseteq V(G) \setminus W$ of size at most $k$
such that $G-W$ admits a proper coloring $f$ that extends $f_w$. Let $B_i^W$ be the set of vertex colored as $i$ by $f_w$, and $B_i$ the vertex colored $i$ by $f$.
Observe that, every vertex $v$ in $B_2 := N_G(B_1^W) \cap (V(G) \setminus W)$ needs either to be colored 2 by $f$ or be deleted (i.e. included in $X$).
The same applies to every vertex $v$ in  $B_1 := N_G(B_2^W) \cap (V(G) \setminus W)$.

Consequently, the task of finding $X$ or the coloring $f$ can be reduced to solving the Annotated Bipartite Coloring of instance $(G-W,B_1,B_2,k)$.
By lemma \ref{lemma:annotated-bipartite-coloring} such instance can be solved in \O{k(n+m)}, using the steps described in the subsection \ref{subsec:the-iterative-compression-technique}
we can affirm that Odd Cycle Transversal can be solved in time \O{3^kkn(n+m)}.

\subsection{Applying OCT to our problem}\label{subsec:applying-oct-to-our-problem}

Back to our problem of $(2,1)$ recognition, we already have either a solution or a $(3,1)$-partition of $G$.
We can see the recognition of if an $(3,1)$-partition is in fact a $(2,1)$-partition as the work of rearranging some vertex from the tri-partition to the clique.
Notice that at most 3 vertices can be moved from the tri-partition to the clique, as if there was a $C4$ in the tri-partition it was not in fact tripartite.
In order to identify the vertices that should be moved to the clique, one can search for a OCT of size at most 3, if no solution can be found than our instance is a no instance, otherwise we have one of the following cases:

\begin{enumerate}
  \item $|OCT|=1$:

      If the oct consists in only one vertex then we can for every vertex on the tri-partition try to move it to the clique.
      This movement behaviours as follows: First we analyze the intersection between the clique and the Not-Neighbourhood of the vertex,
      if the size is over 2 then this vertex can't be moved to the clique as such movement would move a triangle to the bipartition, therefore invalidating it.
      Otherwise move vertex to the clique and the intersection to the bipartition, if it remains bipartite then the Graph is (2,1). If no vertex can be moved then the instance is not (2,1).
  \item $|OCT|=2$:

      If the oct is of size 2 then one edge must be reallocated to the clique.
      As testing every single one of the edges in the tripartition would be above our desired threshold complexity, we need to find the correct candidates to iterate over.
      This solution is described at the section~\ref{sec:tcp}.
  \item $|OCT|=3$:

      In the case of an oct of size 3 a characteristic must be noted, if the tripartition has at least 2 disjoint triangles,
      then the instance is not (2,1) as it would not be possible to reallocate all triangles to the clique.
      Therefore, once found an oct of size 3 at least one of the vertex of the oct must be moved to the clique,
      we apply the restrictions described in the first case to narrow down the vertices tha can be moved to the clique,
      then for each of these vertices we move it to clique and apply the steps described in~\ref{sec:tcp}.
\end{enumerate}

All the steps described in this section can be summarized in the following pseudo code

%% %% @formatter:off
\begin{code}{3-1-to-2-1}
def 3-1-to-2-1(Graph-3-1 G)
  tri = |$G.I_1 \cup G.I_2 \cup G.I_3$|
  oct = Odd-Cycle-Transversal(tri,3)
  case oct is None then
    return None; # There's more then 3 vertices in the tripartite that should move.
  case oct.size = 1 then
    for v in tri.v do # O(n)
      inter = |$\bar{N}(v) \cap G.C$|
      if size(inter) > 2 then
        break loop;
      cl = G.C - inter + v
      bi = tri - v + inter
      if (cl.is_Clique() and bi.is_Bipartite()) then # O(m)
        return new 2-1-patition{
          |$C$|: cl
          |$I_1$|: bi.|$I_1$|,
          |$I_2$|: bi.|$I_2$|
        }
      end
    end
    return None; # There's no vertex that can be moved.
  case oct.size = 2 then
    return Transform2(tri,G.C,G)
  case oct.size = 3 then
    for v in oct do
      inter = |$\bar{N}(v) \cap G.C$|
      if size(inter) > 2 then
        break loop;
      cl = G.C - inter + v
      neoTri = tri - v + inter
      result = Transform2(neoTri,cl,G)
      if result is not None then
        return result;
    end
    return None;
end
\end{code}
%% @formatter:on

\section{The critical paths and its feasible movements}\label{sec:tcp}

At this point we either already has an answer to our recognition problem, or we know that we must reallocate one edge from the tripartition to the clique so that $G$ is a (2,1) instance.
Unfortunately, to try all the possible movements enumerated by the edges in the tripartition is above our threshold, we must then find a solution that can give us the correct edge in time \O{n*m}.

Let us recall partially the strategy taken by Reed in solving the OCT problem.
Once we are at the Disjoint part of the iterative compression technique, a reduction from this problem to the Annotated Bipartite Coloring problem occurs.
At the selection of the separating set $X$ $k$ iterations of the Ford-Fulkerson are applied to determine such set.

Our problem once again will use the OCT solution to move forward.
Notice that the solution raised by the Reed's algorithm is not unique, i.e. there can be more than one set $X$ such that $G-X$ is bipartite,
but when dealing with OCT of size two every solution must exist in the same paths of the Ford-Fulkerson execution, otherwise there would be no solution to the min-cut max-flow problem of size 2.
Our strategy is that instead of looking for the vertices in $X$ we must analyze the paths found in the Ford-Fulkerson iterations that go through these vertices.
For sake of clarity these paths will be called critical paths from now on.

Let $P_1,P_2$ be the critical paths found by the execution aforementioned, we must now find an edge $(u,v)$ such that $u \in P_1$ and $v \in P_2$ that when reallocated to the clique must transform the tripartite in a bipartite.
In order to do so, the remove of these vertices from the Paths must disconnect the source from the target nodes, and the movements made on $\bar{N}(v) \cap C$ and $\bar{N}(u) \cap C$ to the tripartite must not break the bipartition found.

Assume every vertex on $P_1$ and $P_2$ are named uniquely, each vertex also holds a position on its respective path, closer it is to the source smaller is its number.
Our strategy consists of building two tables $Rank_1,Rank_2$. $Rank_i$ shall store information regard the path $P_i$.
The tables are filled as following:
\begin{itemize}
  \item The unique name of a vertex
  \item The \emph{rank} of a vertex, this can be either its position in the path or the furthest position reached by the previous vertex in this path, whichever is bigger.
  \item The furthest this vertex or the previous vertex can reach in this path
\end{itemize}

After this step we must fill the information of the vertices from one path into the other table, in this case the filling is made as following:
\begin{itemize}
  \item The unique name of a vertex
  \item The \emph{rank} of a vertex, this can be either its position in the path or the furthest position reached by the previous vertex in the other path, whichever is bigger.
  \item The furthest this vertex or the previous vertex can reach in the other path
\end{itemize}

Now for each edge $(u,v)$ that have one end in each path we do:

Let $S:= \bar{N_k(u)} \cap \bar{N_k(v)} \cap C$, if $|S|>2$ then iterate using the next edge. Assuming $u \in P_1$ and $v \in P_2$ we can only progress with the edges that meets the following criteria:

\begin{itemize}
  \item The position of $u$ in $P_1$ must be smaller or equal to both its rank in $rank_1$ and $rank_2$, ie.e if this vertex is moved to the clique none of the previous vertex can reach the target by either path.
  \item The position of $v$ in $P_2$ must be smaller or equal to both its rank in $rank_1$ and $rank_2$, ie.e if this vertex is moved to the clique none of the previous vertex can reach the target by either path.
\end{itemize}

Once these criteria are met if $|S|=0$ we have our solution as we must only move the edge $(u,v)$ to the clique.
Otherwise, we must validate that the movement of $S$ to the intended bipartition does not conflict.
In order to do so, we declare four variables for each vertex $z \in S$:

\begin{itemize}
  \item $right_u, left_u$ are respectively the position of the closest vertex to the source and the position of the closest vertex to the target that $z$ can reach in $P_1$
  \item $right_v, left_v$ are respectively the position of the closest vertex to the source and the position of the closest vertex to the target that $z$ can reach in $P_2$
\end{itemize}

Let max\_right\_u be the bigger $right_u$ and min\_left\_u be the smaller $left_u$ in $S$.
Apply the same idea to $right_v$ and $left_v$ raising max\_right\_v and min\_left\_v.
If max\_right\_u is smaller than the position of $u$ in $P_1$ and max\_right\_v is smaller than the position of $v$ in $P_2$ then the movement is allowed and therefore, the instance is $(2,1)$.
On the other hand if min\_left\_u is bigger than the position of $u$ in $P_1$ and min\_left\_v is bigger than the position of $v$ in $P_2$ then the movement is allowed and therefore, the instance is $(2,1)$.
If for every edge one of these conditions are not met, then the instance is not $(2,1)$.

The algorithm described in this section can be summarized by the following pseudo code.
%% @formatter:off
\begin{code}{Transform2}
def Transform2(Tripartite,clique,G)
  |$P_1$|, |$P_2$| = CriticalPaths(Tripartite)
  # Table
  # rank \ aux
  rank1 = Table[string][Int][Int]
  for v in |$V(G[P_1])$| do # Rank 1
    max_out = 0
    for out in |$E^+(v) \in G[P_1]$| do
      max_out = Max(out.position,max_out)
    end # max_out is the closest vertex to T that can be reached by an out edge
    previous = v.prev
    rank = v.position
    if previous is not None then
      # rank is either v position or the foremost
      # vertex that can be reached by the previous vertex
      rank = Max(rank, rank1[previous][1])
      # max_out is the max between the farthest
      # vertex that can be reached by v or by the previous vertex
      max_out = Max(max_out,rank1[previous][1])
    end
    rank1[v][0] = rank
    rank1[v][1] = max_out
  end
  rank2 = Table[string][Int][Int]
  for u in |$V(G[P_2])$| do # Rank 2
    max_out = 0
    for out in |$E^+(u) \in G[P_2]$| do
      max_out = Max(out,max_out)
    end # max_out is the closest vertex to T that can be reached by an out edge
    previous = u.prev
    rank = u.position
    if previous is not None then
      # rank is either u position or the foremost
      # vertex that can be reached by the previous vertex
      rank = Max(rank, rank2[previous][1])
      # max_out is the max between the farthest
      # vertex that can be reached by u or by the previous vertex
      max_out = Max(max_out,rank2[previous][1])
    end
    rank2[u][0] = rank
    rank2[u][1] = max_out
  end
  for v in |$V(G[P_2])$| # Rank 1.2
    max_out = 0
    for out in |$E^+(v) \in G[P_1 \cup v]$| do
      max_out = Max(out,max_out)
    end # max_out is the closest vertex to T that can be reached by an out edge
    previous = v.prev
    rank = v.position
    if previous is not None then
      # rank is either v position or the foremost
      # vertex that can be reached by the previous vertex in $P_1$
      rank = Max(rank, rank1[previous][1])
      # max_out is the max between the farthest
      # vertex that can be reached by v or by the previous vertex
      max_out = Max(max_out,rank1[previous][1])
    end
    rank1[v][0] = rank
    rank1[v][1] = max_out
  end
  for u in |$V(G[P_1])$| # Rank 2.1
    max_out = 0
    for out in |$E^+(u) \in G[P_2 \cup u]$| do
      max_out = Max(out,max_out)
    end # max_out is the closest vertex to T that can be reached by an out edge
    previous = u.prev
    rank = u.position
    if previous is not None then
      # rank is either u position or the foremost
      # vertex that can be reached by the previous vertex in $P_2$
      rank = Max(rank, rank2[previous][1])
      # max_out is the max between the farthest
      # vertex that can be reached by v or by the previous vertex
      max_out = Max(max_out,rank2[previous][1])
    end
    rank2[v][0] = rank
    rank2[v][1] = max_out
  end
  for z in |$V(clique)$| do # for all vertex in the clique
    # take all edges on the tripartite that have an end on z
    for e(z,i) in |$E(Tripartite \cap N[z])$| do
      z.left_u = |$+\infty$|
      z.left_v = |$+\infty$|
      z.right_u = 0
      z.right_v = 0
      # Find which are the closest and farthest vertex in $P_1$ z can reach
      if i in |$V(G[P_1])$| then
        if i.position < z.left_u then
          z.left_u = i.position
        end
        if i.position > z.right_u then
          z.right_u = i.position
        end
      end
      # Find which are the closest and farthest vertex in $P_2$ z can reach
      if i in |$V(G[P_2])$| then
        if i.position < z.left_v then
          z.left_v = i.position
        end
        if i.position > z.right_v then
          z.right_v = i.position
        end
      end
    end
  end
  for |$e(u,v) \in E(G)$| where |$u \in V(G[P_1])$| and |$u \in V(G[P_1])$| do
    # let $\bar{N_k(v)}$ be the not-neighborhood of a vertex in the clique
    S = |$\bar{N_k(u)} \cap \bar{N_k(v)}$|
    if size(S) <=2 then
     if rank1[u][0] <= u.position then
       if rank2[v][0] <= v.position then
         if rank2[u][0] <= v.position and rank1[v][0] <= u.position then
           if size(S) = 0 then
             c = |$clique \cup \{u,v\}$|
             bi = Bipartition_Of(|$Tripartite \setminus \{u,v\}$|)
             return new 2-1-patition{
               |$C$|: c
               |$I_1$|: bi.|$I_1$|,
               |$I_2$|: bi.|$I_2$|
             }
           end
           max_right_u = 0
           max_right_v = 0
           min_left_u = |$+\infty$|
           min_left_v = |$+\infty$|
           c = |$clique \cup \{u,v\} \setminus S$|
           bi = Bipartition_Of(|$Tripartite \cup S \setminus \{u,v\}$|)
           h = new 2-1-patition{
               |$C$|: c
               |$I_1$|: bi.|$I_1$|,
               |$I_2$|: bi.|$I_2$|
             }
           for z in V(S) do
            if z.right_u > max_right_u then
              max_right_u = z.right_u
            end
            if z.right_v > max_right_v then
              max_right_v = z.right_v
            end
            if z.left_v < min_left_v then
              min_left_v = z.left_v
            end
            if z.left_u < min_left_u then
              min_left_u = z.left_u
            end
           end
           if max_right_u < u.position and max_right_v < v.position then
             return h;
           end
           if min_left_u > u.position and min_left_v > v.position then
             return h;
           end
         end
       end
     end
    end
  end
  return None;
end
\end{code}
%% @formatter:on

