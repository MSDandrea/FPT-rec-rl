\chapter{Introduction}

\section{Current results}

In this section we will explore the current state of the complexity analysis as $r$ and $\ell$ grows. As we fullfill the following table we expose the strategies and how those can be used to enlight the more complex results.

The most trivial result is the recognition of the $(1,0)$-graphs, as in order to recognize it we just need to know if $|E(G)| > 0$. Therefore it's complexity is \O{1}.

\begin{table}[h!]
  \caption{Incomplete complexity analysis of the \RL-partite recognition problem}
  \label{}
  \center
  \begin{tabular}{l|*{6}c}
    \toprule
    \backslashbox{$r$}{$\ell$} & 0 & 1 & 2 & 3 & 4 & \ldots \\
    \midrule
    0 & -  & \? & \? & \? & \? & \ldots \\
    1 & \O{1} & \? & \? & \? & \? & \ldots \\
    2 & \? & \? & \? & \? & \? & \ldots \\
    3 & \? & \? & \? & \? & \? & \ldots \\
    4 & \? & \? & \? & \? & \? & \ldots \\
    $\vdots$ & $\vdots$ & $\vdots$ & $\vdots$ & $\vdots$ & $\vdots$ & $\ddots$ \\
  \end{tabular}
\end{table}

\subsection{$m$-bounded results}

Naturally, we wish to find the complexity for those problems of small partition cardinality.

At \cite{konig36}, König showed that it takes \O{m} steps to recognize a bipartite graph. For complete graphs, is enough to check if $|E(G)|=n(n-1)/2$, if it doesn't then we have an answer; If it does, checking vertex by vertex if it's neighborhood contains all other vertex is \O{m}.

For co-bipartite graphs recognition, we need only to verify if it's complement is a bipartite graph. The recognition of a split graph can be done using their vertex degrees \cite{??}, obtaining all vertices degrees is \O{m} therefore the recognition of split graphs is \O{m}
% \begin{table}[h!]
%   \caption{Incomplete complexity analysis of the \RL-partite recognition problem}
%   \label{}
%   \center
%   \begin{tabular}{l|*{6}c}
%     \toprule
%     \backslashbox{$r$}{$\ell$} & 0 & 1 & 2 & 3 & 4 & \ldots \\
%     \midrule
%     0 & -  & \O{m} & \O{m} & \? & \? & \ldots \\
%     1 & \O{1} & \O{m} & \? & \? & \? & \ldots \\
%     2 & \O{m} & \? & \? & \? & \? & \ldots \\
%     3 & \? & \? & \? & \? & \? & \ldots \\
%     4 & \? & \? & \? & \? & \? & \ldots \\
%     $\vdots$ & $\vdots$ & $\vdots$ & $\vdots$ & $\vdots$ & $\vdots$ & $\ddots$ \\
%   \end{tabular}
% \end{table}

\subsection{$NP$-Complete results}

A adequate strategy at this moment is to find when the recognition problem gets $NP$-Complete.

At \cite{gareyjohnson} is shown that 3-coloring a graph (i.e. assign a color between three possibles to each vertex such that no neighborhood repeats a color) is $NP$-Complete, it's trivial to see how the 3-coloring of a graph can be reduced to the problem of finding if a graph is a $(3,0)$-graph, therefore the recognition of $(3,0)$-graphs is $NP$-Complete.

It's noticible that the recognition of \RL-graphs is monotonic, and therefore if the recognition of $(3,0)$-graphs are $NP$-Complete then the recognition of any $(r,0)$-graph or $(3,\ell)$-graph is $NP$-Complete for $r > 3$ and $\ell >0$.

We can extrapolate these findings and argument that the recognition of a $(0,3)$-graph is also $NP$-Complete, as it is the same as recognize it's complement as a $(3,0)$-graph, and use the property of monotonicity to state that the recognition of any $(r,3)$-graph or $(0,\ell)$-graph is $NP$-Complete for $r > 0$ and $\ell >3$.

\begin{table}[h!]
  \caption{Incomplete complexity analysis of the \RL-partite recognition problem}
  \label{}
  \center
  \begin{tabular}{l|*{6}c}
    \toprule
    \backslashbox{$r$}{$\ell$} & 0 & 1 & 2 & 3 & 4 & \ldots \\
    \midrule
    0 & -  & \O{m} & \O{m} & \NPc & \NPc & \ldots \\
    1 & \O{1} & \O{m} & \? & \NPc & \NPc & \ldots \\
    2 & \O{m} & \? & \? & \NPc & \NPc & \ldots \\
    3 & \NPc & \NPc & \NPc & \NPc & \NPc & \ldots \\
    4 & \NPc & \NPc & \NPc & \NPc & \NPc & \ldots \\
    $\vdots$ & $\vdots$ & $\vdots$ & $\vdots$ & $\vdots$ & $\vdots$ & $\ddots$ \\
  \end{tabular}
\end{table}

\subsection{Frontier results}

Finally, the frontier cases $(1,2)$,$(2,1)$ and $(2,2)$ were subject of studies by Brandstädt\cite{brand-84,brand-96}. He's findings show that:
\begin{itemize}
  \item Recognition of $(1,2)$-graphs are \O{n^4}.
  \item Recognition of $(2,1)$-graphs are \O{n^4}.
  \item Recognition of $(2,2)$-graphs are \O{n^{12}}.
\end{itemize}

\begin{table}[h!]
  \caption{Current complexity analysis of the \RL-partite recognition problem}
  \label{table:current-values}
  \center
  \begin{tabular}{l|*{6}c}
    \toprule
    \backslashbox{$r$}{$\ell$} & 0 & 1 & 2 & 3 & 4 & \ldots \\
    \midrule
    0 & -  & \O{m} & \O{m} & \NPc & \NPc & \ldots \\
    1 & \O{1} & \O{m} & \O{n^4} & \NPc & \NPc & \ldots \\
    2 & \O{m} & \O{n^4} & \O{n^{12}} & \NPc & \NPc & \ldots \\
    3 & \NPc & \NPc & \NPc & \NPc & \NPc & \ldots \\
    4 & \NPc & \NPc & \NPc & \NPc & \NPc & \ldots \\
    $\vdots$ & $\vdots$ & $\vdots$ & $\vdots$ & $\vdots$ & $\vdots$ & $\ddots$ \\
  \end{tabular}
\end{table}
