\chapter{Introduction}

\section{Current results}

In this section we will explore the current state of the complexity analysis as $r$ and $\ell$ grows. As we fullfill the following table we expose the strategies and how those can be used to enlight the more complex results.

The most trivial result is the recognition of the $(1,0)$-graphs, as in order to recognize it we just need to know if $|E(G)| > 0$. Therefore it's complexity is \O{1}.

\begin{table}[h!]
  \caption{Incomplete complexity analysis of the \RL-partite recognition problem}
  \label{}
  \center
  \begin{tabular}{l|*{6}c}
    \toprule
    \backslashbox{$r$}{$\ell$} & 0 & 1 & 2 & 3 & 4 & \ldots \\
    \midrule
    0 & -  & \? & \? & \? & \? & \ldots \\
    1 & \O{1} & \? & \? & \? & \? & \ldots \\
    2 & \? & \? & \? & \? & \? & \ldots \\
    3 & \? & \? & \? & \? & \? & \ldots \\
    4 & \? & \? & \? & \? & \? & \ldots \\
    $\vdots$ & $\vdots$ & $\vdots$ & $\vdots$ & $\vdots$ & $\vdots$ & $\ddots$ \\
  \end{tabular}
\end{table}

\subsection{$m$-bounded results}
\begin{table}[h!]
  \caption{Incomplete complexity analysis of the \RL-partite recognition problem}
  \label{}
  \center
  \begin{tabular}{l|*{6}c}
    \toprule
    \backslashbox{$r$}{$\ell$} & 0 & 1 & 2 & 3 & 4 & \ldots \\
    \midrule
    0 & -  & \O{m} & \O{m} & \? & \? & \ldots \\
    1 & \O{1} & \O{m} & \? & \? & \? & \ldots \\
    2 & \O{m} & \? & \? & \? & \? & \ldots \\
    3 & \? & \? & \? & \? & \? & \ldots \\
    4 & \? & \? & \? & \? & \? & \ldots \\
    $\vdots$ & $\vdots$ & $\vdots$ & $\vdots$ & $\vdots$ & $\vdots$ & $\ddots$ \\
  \end{tabular}
\end{table}

\subsection{$NP$-Complete results}
asdjkashjfasdhgdjasd
asdjkasfakjh
asdjkasfakjhasda


asdjkashjfasdhgdjasd

asdjkashjfasdhgdjasd
\begin{table}[h!]
  \caption{Incomplete complexity analysis of the \RL-partite recognition problem}
  \label{}
  \center
  \begin{tabular}{l|*{6}c}
    \toprule
    \backslashbox{$r$}{$\ell$} & 0 & 1 & 2 & 3 & 4 & \ldots \\
    \midrule
    0 & -  & \O{m} & \O{m} & \NPc & \NPc & \ldots \\
    1 & \O{1} & \O{m} & \? & \NPc & \NPc & \ldots \\
    2 & \O{m} & \? & \? & \NPc & \NPc & \ldots \\
    3 & \NPc & \NPc & \NPc & \NPc & \NPc & \ldots \\
    4 & \NPc & \NPc & \NPc & \NPc & \NPc & \ldots \\
    $\vdots$ & $\vdots$ & $\vdots$ & $\vdots$ & $\vdots$ & $\vdots$ & $\ddots$ \\
  \end{tabular}
\end{table}

\subsection{Frontier results}

\begin{table}[h!]
  \caption{Current complexity analysis of the \RL-partite recognition problem}
  \label{table:current-values}
  \center
  \begin{tabular}{l|*{6}c}
    \toprule
    \backslashbox{$r$}{$\ell$} & 0 & 1 & 2 & 3 & 4 & \ldots \\
    \midrule
    0 & -  & \O{m} & \O{m} & \NPc & \NPc & \ldots \\
    1 & \O{1} & \O{m} & \O{n^4} & \NPc & \NPc & \ldots \\
    2 & \O{m} & \O{n^4} & \O{n^{12}} & \NPc & \NPc & \ldots \\
    3 & \NPc & \NPc & \NPc & \NPc & \NPc & \ldots \\
    4 & \NPc & \NPc & \NPc & \NPc & \NPc & \ldots \\
    $\vdots$ & $\vdots$ & $\vdots$ & $\vdots$ & $\vdots$ & $\vdots$ & $\ddots$ \\
  \end{tabular}
\end{table}
